%% From `template_en.tex'.
%% Copyright 2006-1008 Xavier Danaux (xdanaux@gmail.com).
%
% This work may be distributed and/or modified under the
% conditions of the LaTeX Project Public License version 1.3c,
% available at http://www.latex-project.org/lppl/.

\documentclass[11pt,letterpaper]{moderncv}
\moderncvstyle{classic}
\moderncvcolor{blue}
\usepackage[utf8]{inputenc}                   % replace by the encoding you are using
\usepackage[scale=0.8]{geometry}
\usepackage{parskip}

% personal data
\name{Tamara Paula}{Temple}
\title{Webologist and Software Craftsperson}
\phone[mobile]{+1-651-485-2195}
\email{tamara@tamouse.org}
\social[twitter]{tamouse}
\social[github]{tamouse}
\homepage{https://tamouse.org}

%\photo[64pt][0.4pt]{headshotpink.jpg}
%\quote{quote (optional)}

%\nopagenumbers{}


%----------------------------------------------------------------------------------
%            content
%----------------------------------------------------------------------------------
\begin{document}
\maketitle

\section{Summary}
\cvline{}{Working at the intersection of Rails, GraphQL, and React, developing modern web
applications with fast response times. Deeply experienced in software architecture,
design, and development methodologies, especially related to the disciplines of web
applications, user experience, team building, and processes.
}

\section{Skills}
\cvcomputer{general}{business analysis, consulting, facilitating, software engineering, teaching, web development}{software engineering}{APIs, architecture, backend, coding, debugging, frontend, full-stack, GraphQL, low-level design, macro- and microservices, REST, testing, tool chains}
\cvcomputer{libraries + frameworks}{ApolloClient, ApolloServer, GraphQL, jekyll, middleman, Rails, React, Sinatra}{databases}{PostgreSQL, Redis, TimeScaleDB}
\cvcomputer{languages}{CSS3, HTML5, JavaScript, PHP, Perl, Python 3, Ruby, SCSS/SASS, Shell}{servers}{AWS ECS, EBS, S3, Cloudfront, Netlify, nginx, puma, unicorn}
\cvcomputer{tools}{babel, Emacs and Org-mode, make, Miro, Node.js, Notion, npm/npx, rake, RubyMine, Storybook, thor, webpack, Webstorm, yeoman (yo)}{testing}{Jest, Minitest, RSpec, Testing Library}

\cvline{Methods}{\small Agile development, scrum master; Continuous everything -- testing, integration, deployment (CT/CI/CD); Functional Programming; Object-oriented analysis, design, programming, and testing; Precision elicitation and facilitation, communication for action; Process flow analysis, systems theory, cybernetics; Program architecture and design; NLP (Neuro-linguistic Programming) practitioner; Structured analysis, design, development, and testing; Test-/behaviour-driven development; Time series analysis; User-/task-centered design}

\section{Experience}
\cventry{2020-2022}{Staff Software Engineer}{Drip, LLC}{Remote}{}{%
Primarily focused on providing consumer and market analytics, insights, and guidance in the Drip app that enables Drip customers to become more capable email marketers. Acted as a project manager to plan, document, and track projects for the team. Software architecture and design for major features in the insights area, and providing mentoring, direction and guidance to other engineers in Drip.
}
\cvlistitem{\small Developed an architecture for providing recommendations, insights, and proposed next actions to Drip users (i.e. email marketers)}
\cvlistitem{\small Developed a microservice to deliver time scale metrics in order to speed up customer analytics pages}
\cvlistitem{\small Designed an architecture to provide a Guidance System that would be able to take requirements and copy for guidance from any product team and others, and provide it to the front end where needed based on metrices and benchmarks}
\cvlistitem{\small Member of the Diversity, Equity, and Inclusion team at Drip, and led the DEI Affinity Groups project to produce a guide for forming and running affinity groups (i.e. employee resource groups) within Drip}
\cvlistitem{\small Led a team to produce a mentoring and learning environment at Drip, including a guide for mentors and mentees, tools to find or offer mentoring, connecting mentees with mentors, and reporting results}
\cvlistitem{\small In the absense of any project management or leadership in the insights area, acted to perform discovery, planning, discussion, documentation, and create the Epics and Stories for each project, and monitored seeing it through to completion}

\cventry{2017-2020}{API-driven Rubyist, React/GraphQL Maven}{ReachLocal (Gannett), Kickserv Product}{Minneapolis, Minnesota/Remote}{}{%
Software engineering across the entire application stack from the back end in Ruby on Rails with a Postgresql database, implementing a GraphQL server controller with various queries, types, mutations, and resolvers to implement a GraphQL API, and create a React front end that consumes that GraphQL API with Apolloclient in order to replace the legacy JavaScript code from over a decade of various implementation styles and libraries, giving a modern, cohesive approach to front end web application development.
}
\cvlistitem{\small developed the Rails GraphQL backend for the product.}
\cvlistitem{\small architected and developed the JavaScript front end code in React to provide a good separation of concerns ensuring testability and writing tests where none existed prior; lowering maintenance costs, and ensuring easy replaceability to enable new features to be added with less overall effort and more predictability.}
\cvlistitem{\small built Rails models, controllers, views, and React components for new Kickserv features.}
\cvlistitem{\small rebuilt all the tooling for building, testing, and delivering front end code with the Rails application, using modern build technology such as webpack (using webpacker), jest, and the JavaScript Testing Library}

\cventry{2001-2017}{Indepndent Contractor}{Pontiki Software Crafts}{Mendota Heights, Minnesota/Remote}{}{%
Constract software development of web applications, websites, and tools
}
\cvlistitem{\small Clients: Ackmann and Dickenson, Software for Good, Bluewaterbrand, Novu, Shopzilla}

\cventry{1980-present}{Various positions}{Hewlett-Packard (San Jose, Palo Alto)}{}{}{%
Internal Process and Technology Consultant,  Senior Software Engineer/Scientist, Software Quality Engineer manager of 10 engineers, Senior Software Quality Engineer
}
\cvlistitem{\small Career as a software developer, business analyst, process and technology consultant, author, facilitator, trainer}


\section{Education}
\cventry{1975-1980}{Bachelor of Science, Computer Science}{University of Minnesota}{Minneapolis}{\textit{3.2 GPA}}{}

\cvline{Additional Training}{Structured analysis, design, programming and testing, System testing, Objected oriented analysis, design, programming and testing, Consulting and facilitating skills, Neuro-linguistic programming certified practitioner, Project management, Public speaking and presentation skills, You Don't Know JS with Kyle Simpson, Advanced JavaScript, Numerous courses through FrontendMasters.com on React, Webpack, GraphQL, JAMstack, Angular, others}

%\subsection{Miscellaneous}
%\cventry{year--year}{Job title}{Employer}{City}{}{Description line 1\newline{}Description line 2}% arguments 3 to 6 are optional

%\section{Languages}
%\cvlanguage{language 1}{Skill level}{Comment}
%\cvlanguage{language 2}{Skill level}{Comment}
%\cvlanguage{language 3}{Skill level}{Comment}

%\section{Computer skills}
%\cvcomputer{category 1}{XXX, YYY, ZZZ}{category 4}{XXX, YYY, ZZZ}
%\cvcomputer{category 2}{XXX, YYY, ZZZ}{category 5}{XXX, YYY, ZZZ}
%\cvcomputer{category 3}{XXX, YYY, ZZZ}{category 6}{XXX, YYY, ZZZ}

%\section{Interests}
%\cvline{hobby 1}{\small Description}
%\cvline{hobby 2}{\small Description}
%\cvline{hobby 3}{\small Description}

%\renewcommand{\listitemsymbol}{-} % change the symbol for lists

%\section{Extra 1}
%\cvlistitem{Item 1}
%\cvlistitem{Item 2}
%\cvlistitem[+]{Item 3}            % optional other symbol

%\section{Extra 2}
%\cvlistdoubleitem[\Neutral]{Item 1}{Item 4}
%\cvlistdoubleitem[\Neutral]{Item 2}{Item 5}
%\cvlistdoubleitem[\Neutral]{Item 3}{}

% Publications from a BibTeX file
%\nocite{*}
%\bibliographystyle{plain}
%\bibliography{publications}       % 'publications' is the name of a BibTeX file

\end{document}
